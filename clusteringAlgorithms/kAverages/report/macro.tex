\makeatletter

\RequirePackage{multido}
\RequirePackage{ifthen}

% Accessoires pour la macro \ent

\newcounter{@ntier} \newcounter{p@rtie} \newcounter{b@se}
\newcounter{@bs} \newcounter{f@und}

\newcommand{\ms@p}{\hspace{0.15 em}}

\newcommand{\surtr@is}[2]{\ifthenelse{#2>0}{\ifthenelse{#1>99}{#1}{\ifthenelse{#1>9}{0#1}{00#1}}}{#1}}

% \ent{un_entier} permet de formatter automatiquement
% un nombre entier en ins�rant aux endroits ad�quats
% des espaces s�parateurs des milliers. Les compteurs
% d�finis ci-dessus ainsi que les macros \ms@p et
% \surtr@is sont utilis�s par cette macro.
\newcommand{\ent}[1]{%
  \setcounter{@ntier}{#1}%
  \setcounter{b@se}{1000000}%
  \setcounter{@bs}{\theb@se-1}%
  \setcounter{f@und}{0}%
  \mbox{%
    \multido{\i=0+1}{2}{%
      \ifthenelse{\the@ntier>\the@bs}{%
        \setcounter{p@rtie}{\the@ntier/\theb@se}%
        \surtr@is{\thep@rtie}{\thef@und}\ms@p%
        \setcounter{@ntier}{\the@ntier-\thep@rtie*\theb@se}%
        \setcounter{f@und}{1}%
        }{\ifthenelse{\thef@und>0}{000\ms@p}{}}%
      \setcounter{b@se}{\theb@se/1000}%
      \setcounter{@bs}{\theb@se-1}%
      }%
    \surtr@is{\the@ntier}{\thef@und}%
    }%
  }

\newlength{\sp@cesize}

\newcommand{\cname}[3]{%
  \settowidth{\sp@cesize}{~}%
  $<$\,%
  #1~\hspace{-0.5\sp@cesize}/ \hspace{-0.75\sp@cesize}%
  #2~\hspace{-0.5\sp@cesize}/ \hspace{-0.75\sp@cesize}%
  #3%
  \,$>$%
}

\newcommand{\noun}[1]{\textsc{#1}} %raccourci pour \textsc
\newcommand{\ms}[1]{\texttt{#1}} %racccourci pour ``monospaced''

\newcommand{\lat}{\emph} %pour les mots latins
\newcommand{\for}{\emph} %pour les mots �trangers

\newcommand{\cooc}{cooccurrence}
\newcommand{\copres}{copr�sence}
\newcommand{\cm}{classe de mots-cl�s} \newcommand{\cms}{classes de mots-cl�s}
\newcommand{\mc}{mot-cl�} \newcommand{\mcs}{mots-cl�s}
\newcommand{\cvl}{\noun{Chavl}} \newcommand{\cvp}{\ms{chavleps}}
\newcommand{\cad}{---}
\newcommand{\etype}{�cart type}

\newcommand{\mot}[1]{\textsl{#1}} % formatage standard des mots utilis�s comme exemples
\newcommand{\sememe}[1]{\mot{#1}}

%% \newcommand{\b@rsepa}{\hspace{0.015em}}
%% \newcommand{\b@rsepb}{\hspace{0.05em}}

\newcommand{\b@rsepa}{\hspace{0.08em}}
\newcommand{\b@rsepb}{\hspace{0.07em}}

\newcommand{\doubleb@r}{\mbox{/\hspace{-0.25em}/}}

\newcommand{\seme}[1]{\textnormal{/\nolinebreak{}\b@rsepa\nolinebreak{}\textsf{#1}\nolinebreak{}\b@rsepb\nolinebreak{}/\hspace{-0.5mm}}}
\newcommand{\taxeme}[1]{\textnormal{\doubleb@r\nolinebreak{}\b@rsepa\nolinebreak{}\textsf{#1}\nolinebreak{}\b@rsepb\nolinebreak{}\doubleb@r}\hspace{-0.5mm}}

\newcommand{\gl}[1]{<<\,#1\,>>} % Guillemets standard

\newlength{\boxdecal}

\newenvironment{todo}{%
  \begin{center}%
    \begin{minipage}{\textwidth-2cm}%
      \itshape\color{red}%
}{%
  \end{minipage}\end{center}%
}

\newcounter{si@lreadyUsed}
\setcounter{si@lreadyUsed}{0}

\newcommand{\si}{%
  \ifthenelse{\thesi@lreadyUsed=0}{%
    \setcounter{si@lreadyUsed}{1}%
    S�mantique interpr�tative}{%
    s�mantique interpr�tative%
  }%
}

\newcommand{\sd}{s�mantique diff�rentielle}

\newcommand{\cetadir}{c'est-�-dire}

\newlength{\quoteWidth}

\RequirePackage{xspace}

\newcommand{\faestos}{\textsc{Faestos}\xspace}


% bibliography

\def\thebibliography#1{\section*{References}
  \global\def\@listi{\leftmargin\leftmargini
               \labelwidth\leftmargini \advance\labelwidth-\labelsep
               \topsep 1pt plus 2pt minus 1pt
               \parsep 0.25ex plus 1pt \itemsep 0.25ex plus 1pt}
  \list {[\arabic{enumi}]}{\settowidth\labelwidth{[#1]}\leftmargin\labelwidth
    \advance\leftmargin\labelsep\usecounter{enumi}}
    \def\newblock{\hskip .11em plus .33em minus -.07em}
    \sloppy
    \sfcode`\.=1000\relax}

\def\@up#1{\raise.2ex\hbox{#1}}

% most of cite format is from aclsub.sty by SMS

% don't box citations, separate with ; and a space
% also, make the penalty between citations negative: a good place to break
% changed comma back to semicolon pj 2/1/90
% \def\@citex[#1]#2{\if@filesw\immediate\write\@auxout{\string\citation{#2}}\fi
% \def\@citea{}\@cite{\@for\@citeb:=#2\do
%   {\@citea\def\@citea{;\penalty\@citeseppen\ }\@ifundefined
%      {b@\@citeb}{{\bf ?}\@warning
%      {Citation `\@citeb' on page \thepage \space undefined}}%
% {\csname b@\@citeb\endcsname}}}{#1}}

% don't box citations, separate with ; and a space
% Replaced for multiple citations (pj) 
% don't box citations and also add space, semicolon between multiple citations
\def\@citex[#1]#2{\if@filesw\immediate\write\@auxout{\string\citation{#2}}\fi
  \def\@citea{}\@cite{\@for\@citeb:=#2\do
     {\@citea\def\@citea{; }\@ifundefined
       {b@\@citeb}{{\bf ?}\@warning
        {Citation `\@citeb' on page \thepage \space undefined}}%
 {\csname b@\@citeb\endcsname}}}{#1}}

% Allow short (name-less) citations, when used in
% conjunction with a bibliography style that creates labels like
%       \citename{<names>, }<year>
% 
\let\@internalcite\cite
\def\cite{\def\citename##1{##1, }\@internalcite}
\def\shortcite{\def\citename##1{}\@internalcite}
\def\newcite{\def\citename##1{{\frenchspacing##1} (}\@internalciteb}

% Macros for \newcite, which leaves name in running text, and is
% otherwise like \shortcite.
\def\@citexb[#1]#2{\if@filesw\immediate\write\@auxout{\string\citation{#2}}\fi
  \def\@citea{}\@newcite{\@for\@citeb:=#2\do
    {\@citea\def\@citea{;\penalty\@m\ }\@ifundefined
       {b@\@citeb}{{\bf ?}\@warning
       {Citation `\@citeb' on page \thepage \space undefined}}%
{\csname b@\@citeb\endcsname}}}{#1}}
\def\@internalciteb{\@ifnextchar [{\@tempswatrue\@citexb}{\@tempswafalse\@citexb[]}}

\def\@newcite#1#2{{#1\if@tempswa, #2\fi)}}

\def\@biblabel#1{\def\citename##1{##1}[#1]\hfill}

%%% More changes made by SMS (originals in latex.tex)
% Use parentheses instead of square brackets in the text.
\def\@cite#1#2{({#1\if@tempswa , #2\fi})}


% Don't put a label in the bibliography at all.  Just use the unlabeled format
% instead.
\def\thebibliography#1{\vskip\parskip%
\vskip\baselineskip%
\def\baselinestretch{1}%
\ifx\@currsize\normalsize\@normalsize\else\@currsize\fi%
\vskip-\parskip%
\vskip-\baselineskip%
\section*{References\@mkboth
 {References}{References}}\list
 {}{\setlength{\labelwidth}{0pt}\setlength{\leftmargin}{\parindent}
 \setlength{\itemindent}{-\parindent}}
 \def\newblock{\hskip .11em plus .33em minus -.07em}
 \sloppy\clubpenalty4000\widowpenalty4000
 \sfcode`\.=1000\relax}
\let\endthebibliography=\endlist

% Allow for a bibliography of sources of attested examples
\def\thesourcebibliography#1{\vskip\parskip%
\vskip\baselineskip%
\def\baselinestretch{1}%
\ifx\@currsize\normalsize\@normalsize\else\@currsize\fi%
\vskip-\parskip%
\vskip-\baselineskip%
\section*{Sources of Attested Examples\@mkboth
 {Sources of Attested Examples}{Sources of Attested Examples}}\list
 {}{\setlength{\labelwidth}{0pt}\setlength{\leftmargin}{\parindent}
 \setlength{\itemindent}{-\parindent}}
 \def\newblock{\hskip .11em plus .33em minus -.07em}
 \sloppy\clubpenalty4000\widowpenalty4000
 \sfcode`\.=1000\relax}
\let\endthesourcebibliography=\endlist

\def\@lbibitem[#1]#2{\item[]\if@filesw 
      { \def\protect##1{\string ##1\space}\immediate
        \write\@auxout{\string\bibcite{#2}{#1}}\fi\ignorespaces}}

\def\@bibitem#1{\item\if@filesw \immediate\write\@auxout
       {\string\bibcite{#1}{\the\c@enumi}}\fi\ignorespaces}


\makeatother
