\documentclass[a4paper,twoside]{article}
\usepackage{amsmath,amsthm,amssymb}
\usepackage[T1]{fontenc}
\usepackage[latin1]{inputenc}
\usepackage[dvips]{graphicx}
\usepackage[french]{babel}

\makeatletter

\RequirePackage{multido}
\RequirePackage{ifthen}

% Accessoires pour la macro \ent

\newcounter{@ntier} \newcounter{p@rtie} \newcounter{b@se}
\newcounter{@bs} \newcounter{f@und}

\newcommand{\ms@p}{\hspace{0.15 em}}

\newcommand{\surtr@is}[2]{\ifthenelse{#2>0}{\ifthenelse{#1>99}{#1}{\ifthenelse{#1>9}{0#1}{00#1}}}{#1}}

% \ent{un_entier} permet de formatter automatiquement
% un nombre entier en ins�rant aux endroits ad�quats
% des espaces s�parateurs des milliers. Les compteurs
% d�finis ci-dessus ainsi que les macros \ms@p et
% \surtr@is sont utilis�s par cette macro.
\newcommand{\ent}[1]{%
  \setcounter{@ntier}{#1}%
  \setcounter{b@se}{1000000}%
  \setcounter{@bs}{\theb@se-1}%
  \setcounter{f@und}{0}%
  \mbox{%
    \multido{\i=0+1}{2}{%
      \ifthenelse{\the@ntier>\the@bs}{%
        \setcounter{p@rtie}{\the@ntier/\theb@se}%
        \surtr@is{\thep@rtie}{\thef@und}\ms@p%
        \setcounter{@ntier}{\the@ntier-\thep@rtie*\theb@se}%
        \setcounter{f@und}{1}%
        }{\ifthenelse{\thef@und>0}{000\ms@p}{}}%
      \setcounter{b@se}{\theb@se/1000}%
      \setcounter{@bs}{\theb@se-1}%
      }%
    \surtr@is{\the@ntier}{\thef@und}%
    }%
  }

\newlength{\sp@cesize}

\newcommand{\cname}[3]{%
  \settowidth{\sp@cesize}{~}%
  $<$\,%
  #1~\hspace{-0.5\sp@cesize}/ \hspace{-0.75\sp@cesize}%
  #2~\hspace{-0.5\sp@cesize}/ \hspace{-0.75\sp@cesize}%
  #3%
  \,$>$%
}

\newcommand{\noun}[1]{\textsc{#1}} %raccourci pour \textsc
\newcommand{\ms}[1]{\texttt{#1}} %racccourci pour ``monospaced''

\newcommand{\lat}{\emph} %pour les mots latins
\newcommand{\for}{\emph} %pour les mots �trangers

\newcommand{\cooc}{cooccurrence}
\newcommand{\copres}{copr�sence}
\newcommand{\cm}{classe de mots-cl�s} \newcommand{\cms}{classes de mots-cl�s}
\newcommand{\mc}{mot-cl�} \newcommand{\mcs}{mots-cl�s}
\newcommand{\cvl}{\noun{Chavl}} \newcommand{\cvp}{\ms{chavleps}}
\newcommand{\cad}{---}
\newcommand{\etype}{�cart type}

\newcommand{\mot}[1]{\textsl{#1}} % formatage standard des mots utilis�s comme exemples
\newcommand{\sememe}[1]{\mot{#1}}

%% \newcommand{\b@rsepa}{\hspace{0.015em}}
%% \newcommand{\b@rsepb}{\hspace{0.05em}}

\newcommand{\b@rsepa}{\hspace{0.08em}}
\newcommand{\b@rsepb}{\hspace{0.07em}}

\newcommand{\doubleb@r}{\mbox{/\hspace{-0.25em}/}}

\newcommand{\seme}[1]{\textnormal{/\nolinebreak{}\b@rsepa\nolinebreak{}\textsf{#1}\nolinebreak{}\b@rsepb\nolinebreak{}/\hspace{-0.5mm}}}
\newcommand{\taxeme}[1]{\textnormal{\doubleb@r\nolinebreak{}\b@rsepa\nolinebreak{}\textsf{#1}\nolinebreak{}\b@rsepb\nolinebreak{}\doubleb@r}\hspace{-0.5mm}}

\newcommand{\gl}[1]{<<\,#1\,>>} % Guillemets standard

\newlength{\boxdecal}

\newenvironment{todo}{%
  \begin{center}%
    \begin{minipage}{\textwidth-2cm}%
      \itshape\color{red}%
}{%
  \end{minipage}\end{center}%
}

\newcounter{si@lreadyUsed}
\setcounter{si@lreadyUsed}{0}

\newcommand{\si}{%
  \ifthenelse{\thesi@lreadyUsed=0}{%
    \setcounter{si@lreadyUsed}{1}%
    S�mantique interpr�tative}{%
    s�mantique interpr�tative%
  }%
}

\newcommand{\sd}{s�mantique diff�rentielle}

\newcommand{\cetadir}{c'est-�-dire}

\newlength{\quoteWidth}

\RequirePackage{xspace}

\newcommand{\faestos}{\textsc{Faestos}\xspace}


% bibliography

\def\thebibliography#1{\section*{References}
  \global\def\@listi{\leftmargin\leftmargini
               \labelwidth\leftmargini \advance\labelwidth-\labelsep
               \topsep 1pt plus 2pt minus 1pt
               \parsep 0.25ex plus 1pt \itemsep 0.25ex plus 1pt}
  \list {[\arabic{enumi}]}{\settowidth\labelwidth{[#1]}\leftmargin\labelwidth
    \advance\leftmargin\labelsep\usecounter{enumi}}
    \def\newblock{\hskip .11em plus .33em minus -.07em}
    \sloppy
    \sfcode`\.=1000\relax}

\def\@up#1{\raise.2ex\hbox{#1}}

% most of cite format is from aclsub.sty by SMS

% don't box citations, separate with ; and a space
% also, make the penalty between citations negative: a good place to break
% changed comma back to semicolon pj 2/1/90
% \def\@citex[#1]#2{\if@filesw\immediate\write\@auxout{\string\citation{#2}}\fi
% \def\@citea{}\@cite{\@for\@citeb:=#2\do
%   {\@citea\def\@citea{;\penalty\@citeseppen\ }\@ifundefined
%      {b@\@citeb}{{\bf ?}\@warning
%      {Citation `\@citeb' on page \thepage \space undefined}}%
% {\csname b@\@citeb\endcsname}}}{#1}}

% don't box citations, separate with ; and a space
% Replaced for multiple citations (pj) 
% don't box citations and also add space, semicolon between multiple citations
\def\@citex[#1]#2{\if@filesw\immediate\write\@auxout{\string\citation{#2}}\fi
  \def\@citea{}\@cite{\@for\@citeb:=#2\do
     {\@citea\def\@citea{; }\@ifundefined
       {b@\@citeb}{{\bf ?}\@warning
        {Citation `\@citeb' on page \thepage \space undefined}}%
 {\csname b@\@citeb\endcsname}}}{#1}}

% Allow short (name-less) citations, when used in
% conjunction with a bibliography style that creates labels like
%       \citename{<names>, }<year>
% 
\let\@internalcite\cite
\def\cite{\def\citename##1{##1, }\@internalcite}
\def\shortcite{\def\citename##1{}\@internalcite}
\def\newcite{\def\citename##1{{\frenchspacing##1} (}\@internalciteb}

% Macros for \newcite, which leaves name in running text, and is
% otherwise like \shortcite.
\def\@citexb[#1]#2{\if@filesw\immediate\write\@auxout{\string\citation{#2}}\fi
  \def\@citea{}\@newcite{\@for\@citeb:=#2\do
    {\@citea\def\@citea{;\penalty\@m\ }\@ifundefined
       {b@\@citeb}{{\bf ?}\@warning
       {Citation `\@citeb' on page \thepage \space undefined}}%
{\csname b@\@citeb\endcsname}}}{#1}}
\def\@internalciteb{\@ifnextchar [{\@tempswatrue\@citexb}{\@tempswafalse\@citexb[]}}

\def\@newcite#1#2{{#1\if@tempswa, #2\fi)}}

\def\@biblabel#1{\def\citename##1{##1}[#1]\hfill}

%%% More changes made by SMS (originals in latex.tex)
% Use parentheses instead of square brackets in the text.
\def\@cite#1#2{({#1\if@tempswa , #2\fi})}


% Don't put a label in the bibliography at all.  Just use the unlabeled format
% instead.
\def\thebibliography#1{\vskip\parskip%
\vskip\baselineskip%
\def\baselinestretch{1}%
\ifx\@currsize\normalsize\@normalsize\else\@currsize\fi%
\vskip-\parskip%
\vskip-\baselineskip%
\section*{References\@mkboth
 {References}{References}}\list
 {}{\setlength{\labelwidth}{0pt}\setlength{\leftmargin}{\parindent}
 \setlength{\itemindent}{-\parindent}}
 \def\newblock{\hskip .11em plus .33em minus -.07em}
 \sloppy\clubpenalty4000\widowpenalty4000
 \sfcode`\.=1000\relax}
\let\endthebibliography=\endlist

% Allow for a bibliography of sources of attested examples
\def\thesourcebibliography#1{\vskip\parskip%
\vskip\baselineskip%
\def\baselinestretch{1}%
\ifx\@currsize\normalsize\@normalsize\else\@currsize\fi%
\vskip-\parskip%
\vskip-\baselineskip%
\section*{Sources of Attested Examples\@mkboth
 {Sources of Attested Examples}{Sources of Attested Examples}}\list
 {}{\setlength{\labelwidth}{0pt}\setlength{\leftmargin}{\parindent}
 \setlength{\itemindent}{-\parindent}}
 \def\newblock{\hskip .11em plus .33em minus -.07em}
 \sloppy\clubpenalty4000\widowpenalty4000
 \sfcode`\.=1000\relax}
\let\endthesourcebibliography=\endlist

\def\@lbibitem[#1]#2{\item[]\if@filesw 
      { \def\protect##1{\string ##1\space}\immediate
        \write\@auxout{\string\bibcite{#2}{#1}}\fi\ignorespaces}}

\def\@bibitem#1{\item\if@filesw \immediate\write\@auxout
       {\string\bibcite{#1}{\the\c@enumi}}\fi\ignorespaces}


\makeatother


\begin{document}

\title{k-averages: criterions, convergence}

\author{Mathias Rossignol}

\maketitle

\section{Conventions, possible objective functions}

Partition of $n$ objects $\mathcal{O} = \left\{o_1, \ldots,
o_n\right\}$ into $k$ classes $\mathcal{C} = \left\{c_1, \ldots,
c_k\right\}$. We'll write $n_{c_i}$ the cardinal of class $c_i$, and
$o_{ij}$ its elements: $c_i = \left\{o_{i1}, \ldots, o_{in_{c_i}}\right\}$.

To simplify below, when we're simply considering one class, no matter
which, we shall omit the first index and write $c = \left\{o_1,
\ldots, o_{n_c}\right\}$.

The similarity between objects shall be written $s\left(o_i, o_j\right)$.

What we're interested in: maximising the average intra-class
object-to-object similarity. That's pretty straightforward, as far as
I can tell there's only one possible variant: whether we have a
class-by-class normalization or only a global one.

In the first case, objects appearing in small classes will have more
\gl{weight} in the function. In the second, big classes will tend to
dominate the objective function. So: neither is 100\'\%
satisfying. Better try to implement both and test.

Note that since the number of obejcts does not vary, global
normalization (divide by number of objects) is equivalent to no
normalization at all.

Other possible variants: using other averaging methods (geometric
mean? Does that make sense?). Maybe later.

\section{Impact of object reallocation on class quality}

Let us extend the notation $s$ to the \emph{similarity of an object to a
  class}, which we define as the average similarity of that object
with all objects of the class. $s(o,c)$ accepts two definitions,
depending on whether or not $o$ is in $c$:

If $o \notin c$,
\begin{equation}
  s\left(o,c\right) = \frac{1}{n_c} \sum_{i=1}^{n_c}s\left(o, o_i\right)
\end{equation}

If $o \in c$, then necessarily $\exists i \mid o = o_i$
\begin{equation} \label{soc_inclass}
  s\left(o,c\right) = s\left(o_i, c\right) = \frac{1}{n_c-1} \sum_{j=1 \ldots n_c, j \neq i} s\left(o_i, o_j\right)
\end{equation}

Let's call \gl{quality} of a class the average intra-class object-to-object similarity, and write it $\mathcal{Q}$:
\begin{equation}
\mathcal{Q}\left(c\right) = \frac{1}{n_c} \sum_{i=1}^{n_c} s\left(o_i, c\right)
\end{equation}

Since all objects are in $c$, we use the formula in (\ref{soc_inclass}) to get:
\begin{equation}
  \begin{aligned}
    \mathcal{Q}\left(c\right) & = \frac{1}{n_c} \sum_{i=1}^{n_c} \frac{1}{n_c-1} \sum_{\substack{j=1 \ldots n_c\\j \neq i}} s\left(o_i, o_j\right) \\
                              & = \frac{1}{n_c(n_c-1)} \sum_{i=1}^{n_c} \sum_{\substack{j=1 \ldots n_c\\j \neq i}} s\left(o_i, o_j\right)
  \end{aligned}
\end{equation}

Using the assumption that the similarity matrix is symmetrical, we can reach (this is an indispensable transformation for future calculations):
\begin{equation}
    \mathcal{Q}\left(c\right) = \frac{2}{n_c(n_c-1)} \sum_{i=2}^{n_c} \sum_{j=1}^{i-1} s\left(o_i, o_j\right)
\end{equation}

For future use, let us define the notation:
\begin{equation}
  \Sigma(c) = \sum_{i=2}^{n_c} \sum_{j=1}^{i-1} s\left(o_i, o_j\right)
\end{equation}

Thus:
\begin{equation}
    \mathcal{Q}\left(c\right) = \frac{2}{n_c(n_c-1)}\Sigma(c) \phantom{XX}\mathrm{and}\phantom{XX} \Sigma(c) = \frac{n_c(n_c-1)\mathcal{Q}\left(c\right)}{2}
\end{equation}


\subsection{Removing an object from a class}

Assuming that $o \in c$, necessarily $\exists i \mid o=o_i$. Since the
numbering of objects is arbitrary, we can assume that $o = o_{n_c}$
then generalize from the result thus obtained.

\begin{equation}
  \begin{aligned}
    \mathcal{Q}\left(c \smallsetminus o_{n_c}\right) & = \frac{2}{(n_c-1)(n_c-2)} \sum_{i=2}^{n_c-1} \sum_{j=1}^{i-1} s\left(o_i, o_j\right) \\
                                                   & = \frac{2}{(n_c-1)(n_c-2)} \left[\Sigma(c) - \sum_{j=1}^{n_c-1} s\left(o_{n_c}, o_j\right) \right] \\
                                                   & = \frac{2}{(n_c-1)(n_c-2)} \left[\Sigma(c) - (n_c-1)s\left(o_{n_c}, c\right) \right] \\
                                                   & = \frac{2n_c(n_c-1)\mathcal{Q}(c)}{2(n_c-1)(n_c-2)} - \frac{2(n_c-1)s\left(o_{n_c}, c\right)}{(n_c-1)(n_c-2)}\\
                                                   & = \frac{n_c \mathcal{Q}(c)  - 2s\left(o_{n_c}, c\right)}{n_c-2}
  \end{aligned}
\end{equation}

The quality of a class after removal of an object is thus:

\begin{equation}
  \mathcal{Q}\left(c \smallsetminus o\right) = \frac{n_c \mathcal{Q}(c)  - 2s\left(o, c\right)}{n_c-2}
\end{equation}

And the change in quality from its previous value:

\begin{equation} \label{deltaRemove}
  \begin{aligned}
    \mathcal{Q}\left(c \smallsetminus o\right) - \mathcal{Q}\left(c\right) & = \frac{n_c \mathcal{Q}(c)  - (n_c-2) \mathcal{Q}(c)  - 2s\left(o, c\right)}{n_c-2} \\
                                                                           & = \frac{2\left( \mathcal{Q}(c) - s\left(o, c\right)\right)}{n_c-2}
    \end{aligned}
\end{equation}


\subsection{Adding an object to a class}

Assuming that $o \notin c$, we'll consider for the sake of simplicity that $o$ becomes $o_{n_c+1}$ in the modified class $c$. Following a path similar to above, we get:
\begin{equation}
  \begin{aligned}
    \mathcal{Q}(c \cup o_{n_c+1}) & = \frac{2}{n_c(n_c+1)} \sum_{i=2}^{n_c+1} \sum_{j=1}^{i-1} s\left(o_i, o_j\right) \\
                                & = \frac{2}{n_c(n_c+1)} \left[\Sigma(c) + n_c s\left(o_{n_c+1}, c\right)\right] \\
                                & = \frac{(n_c-1) \mathcal{Q}(c)  + 2s\left(o_{n_c+1}, c\right)}{n_c+1}
  \end{aligned}
\end{equation}

The quality of a class $c$ after adding an object $o$ is thus:

\begin{equation}
  \mathcal{Q}\left(c \cup o\right) = \frac{(n_c-1) \mathcal{Q}(c)  + 2s\left(o, c\right)}{n_c+1}
\end{equation}

And the change in quality from its previous value:

\begin{equation} \label{deltaAdd}
  \begin{aligned}
    \mathcal{Q}\left(c \cup o\right) - \mathcal{Q}\left(c\right) & = \frac{(n_c-1) \mathcal{Q}(c)  - (n_c+1) \mathcal{Q}(c)  + 2s\left(o, c\right)}{n_c+1} \\
                                                                           & = \frac{2\left(s\left(o, c\right)-\mathcal{Q}(c)\right)}{n_c+1}
    \end{aligned}
\end{equation}




\section{Impact of object reallocation on global quality (objective function)}

\subsection{Class-normalized objective function}

In that case, the calculation is direct: from (\ref{deltaRemove}) and
(\ref{deltaAdd}), we can see that the impact on the objective function
of moving an object $c$ from class $c_s$ (``source''), to whom it
belongs, to a distinct class $c_t$ (``target'') is:

\begin{equation}
  \delta = \frac{2\left(s\left(o, c_t\right)-\mathcal{Q}(c_t)\right)}{n_{c_t}+1} + \frac{2\left( \mathcal{Q}(c_s) - s\left(o, c_s\right)\right)}{n_{c_s}-2}
\end{equation}

Using this value as the basis to decide object reallocation, and only
performing reallocations while it is strictly positive, ensures that
the objective function is strictly increasing, and therefore that the
process converges to one of its local maxima.

\subsection{Object-normalized objective function}

This complicates the calculation a bit, but not much: when moving an
object $c$ from class $c_s$ (``source''), to whom it belongs, to a
distinct class $c_t$ (``target''), $(n_{c_s}-1)$ objects are affected
by the variation in (\ref{deltaRemove}), and $n_{c_t})$ are affected
by that in (\ref{deltaAdd}), in addition to the variation in simiarity
of $o$ to the class it belongs to:

\begin{equation}
  \delta = \frac{2n_{c_t}\left(s\left(o, c_t\right)-\mathcal{Q}(c_t)\right)}{n_{c_t}+1} + \frac{2(n_{c_s}-1)\left( \mathcal{Q}(c_s) - s\left(o, c_s\right)\right)}{n_{c_s}-2} + s(o,c_t) - s(o,c_s)
\end{equation}

Once again, using this value as the basis to decide object
reallocation, and only performing reallocations while it is strictly
positive, ensures that the objective function is strictly increasing,
and therefore that the process converges to one of its local maxima.


\end{document}
